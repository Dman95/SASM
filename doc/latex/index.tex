\hypertarget{index_intro_sec}{}\section{Introduction}\label{index_intro_sec}
Below you will find what and where to modify appropriate header and source files when you want to add a feature.\hypertarget{index_section1}{}\section{Commenting}\label{index_section1}
When commenting use doxygen's syntax. Place the comment above the intended line. After adding the feature, run doxygen.\+exe configfile to update the documentation. Be sure to double check the documentation to ensure the added comments were parsed correctly.\hypertarget{index_section2}{}\section{Adding Assembler Support}\label{index_section2}
Adding support is a relatively straight forward process. Each supported assembler is a derived \hyperlink{class_assembler}{Assembler} and has its own header and cpp file.\hypertarget{index_step1}{}\subsection{Step 1\+: Creating the header and cpp}\label{index_step1}
The first step is to create the new header and cpp file for the assembler. These should be named all lowercase without gaps\hypertarget{index_step2}{}\subsection{Step 2\+: Creating the assembler class}\label{index_step2}
The decleration of the class may best be discussed in light of already supported assemblers. Take for example, \hyperlink{class_n_a_s_m}{N\+A\+S\+M}. The \hyperlink{class_n_a_s_m}{N\+A\+S\+M} class is defined as\+: class \hyperlink{class_n_a_s_m}{N\+A\+S\+M} \+: public \hyperlink{class_assembler}{Assembler}

The generic definition is class Y\+O\+U\+R\+A\+S\+S\+E\+M\+B\+L\+E\+R \+: public \hyperlink{class_assembler}{Assembler}

The variables and methods of Y\+O\+U\+R\+A\+S\+S\+E\+M\+B\+L\+E\+R should be the virtual methods of \hyperlink{class_assembler}{Assembler}. If you are unsure what to add, refer to the already supported assemler classes. You may copy and paste them.\hypertarget{index_step3}{}\subsection{Step 3\+: Adding it to the Build Options}\label{index_step3}
You should hopefully know enough Q\+T to be able to add form controls. The code for modifying the build menu can be found in \hyperlink{mainwindow_8cpp}{mainwindow.\+cpp}. Refer to its documentation for more reference on where to add/modify code.\hypertarget{index_section3}{}\section{Adding Language Support}\label{index_section3}
\hypertarget{index_Help}{}\subsection{File}\label{index_Help}
The help file, \+\_\+\+\_\+.\+xxx, needs to be translated and saved into a new xx.\+xxx file. The void \hyperlink{class_main_window_a4037900bbe42daa151e96ba5c96c8f62}{Main\+Window\+::open\+Help()} must be modified to support the added language. This is done by adding another if statement.\hypertarget{index_startText}{}\subsection{start\+Text}\label{index_startText}
The default assembler skeleton needs to have its comments translated. 